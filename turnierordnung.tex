\documentclass[fontsize=12pt, paper=a4, ngerman]{article}
\usepackage[ngerman]{babel}
\usepackage[utf8]{inputenc}
\usepackage[T1,EU1]{fontenc}

\usepackage{lmodern}
%\usepackage{a4wide}
\usepackage[left=3cm, right=2.5cm, top=3cm, bottom=3cm]{geometry}

\usepackage{enumitem}

\usepackage[hidelinks]{hyperref}
\hypersetup{
  linktoc=all
}

\setlength{\parindent}{0em}

\begin{document}

\author{ Berliner Schachverband e.V. } \title{Turnierordnung} \date{\today} \maketitle

\tableofcontents

\newpage

\section*{Vorwort}
Die Turnierordnung (im Folgenden TO) regelt alle Turniere des Berliner Schachverbandes e.V. (im Folgenden BSV).

Der Landesjugendwart, der Referent für Frauenschach und der Referent für Seniorenschach regeln ihren
Spielbetrieb entsprechend den allgemeinen Bestimmungen dieser Turnierordnung in eigener
Verantwortung.

\section{Allgemeine Bestimmungen}

\subsection{Allgemeine Spielregeln, Verhalten der Spieler}

\begin{enumerate}
\item Die Regeln des Weltschachbundes (FIDE) sind Bestandteil dieser TO.
\item Der BSV ist einer der Trägerverbände der gemeinsamen Turniere der\linebreak Norddeutschen Landesverbände (NDLV), deren Turnierordnung er anerkennt.
\item Bei Einzelkämpfen und an den einzelnen Brettern eines Mannschafts\-kampfes werden ein Sieg mit 1, ein Remis mit 0,5 und ein Verlust mit 0 Punkten gewertet.
Ein kampfloser Einzelsieg (+) wird mit einem Punkt und ein kampfloser Einzelverlust (-) mit 0 Punkten gewertet
\item Alle gespielten Turnierpartien werden entsprechend der Ausschreibungen DWZ- oder ELO-ausgewertet, auch wenn Ranglisten korrigiert werden müssen,
weil Spieler oder Mannschaften vorzeitig ausgeschieden sind oder Spieler am falschen Brett saßen.
\item Die Mannschaft, die mehr Brettpunkte erzielt hat, gewinnt den Wettkampf. Bei Mannschaftskämpfen wird ein Sieg mit 2, ein Unentschieden mit 1 und eine
Niederlage mit 0 Punkten gewertet. Das Verfahren bei Gleichstand zwischen Spielern oder Mannschaften regelt die jeweilige Turnierausschreibung, sofern diese
TO nicht eine ausdrückliche Bestimmung dazu enthält.
\item Die Wartezeit für BSV-Turniere beträgt 30 min. Der zuständige Turnierleiter/Schiedsrichter kann in begründeten Ausnahmefällen die Wartezeit verändern.
\item Das Mitbringen eines ausgeschalteten Mobiltelefons zum Spielort ist gestattet.
\item Dem Spieler ist es untersagt, das Turnierareal alkoholisiert zu betreten bzw. dort alkoholische Getränke zu konsumieren.
Der Spielbereich darf -- auch von Gästen -- nicht mit alkoholischen Getränken betreten werden.
\end{enumerate}

\subsection{Spielbetrieb}

\begin{enumerate}
\item In Verantwortung des Landesspielleiters veranstaltet der BSV jährlich folgende Turniere:
  \begin{enumerate}[label=\alph*.]
  \item Berliner Einzelmeisterschaft (BEM)
  \item Berliner Mannschaftsmeisterschaft (BMM)
  \item Berliner Feierabendliga (BFL)
  \item Berliner Pokal-Einzelmeisterschaft (BPEM)
  \item Berliner Pokal-Mannschaftsmeisterschaft (BPMM)
  \item Berliner Schnellschach-Einzelmeisterschaft (BSEM)
  \item Berliner Schnellschach-Mannschaftsmeisterschaft (BSMM)
  \item Berliner Blitz-Einzelmeisterschaft (BBEM)
  \item Berliner Blitz-Mannschaftsmeisterschaft (BBMM)
  \end{enumerate}
  Er kann diese Turniere -- mit Ausnahme der BMM -- als offene Turniere ausschreiben und weitere Turniere durchführen.
\item Der Landesspielleiter ist Hauptturnierleiter für alle Turniere. Er kann für einzelne Turniere weitere Turnierleiter und Schiedsrichter einsetzen.
\item Die Postadresse des Landesspielleiters und aller von ihm eingesetzten Turnierleiter ist die Geschäftsadresse des BSV.
\item Turnierausschreibungen sollen mindestens vier Wochen vor dem in der jeweiligen Ausschreibung genannten Meldeschluss den Vereinen zugesendet und auf der
Homepage des BSV veröffentlicht werden. Turnierausschreibungen müssen mindestens enthalten:
  \begin{enumerate}[label=\alph*.]
  \item Art der Veranstaltung und Austragungsmodus
  \item Spielort, Spieltage, Spielbeginn und Bedenkzeit
  \item Turnierleitung und Schiedsrichter
  \item Teilnahmebedingungen und Anmeldemodalitäten
  \item Startgeld, erreichbare Ziele und Preise
  \item Höhe des Reuegeldes
  \end{enumerate}
\item Alle Turniere werden bei den dem Verband angeschlossenen Vereinen durchgeführt. Der Landesspielleiter kann zu jedem Turnier oder für einzelne Runden
einen neutralen Veranstaltungsort und einen abweichenden Spieltermin festlegen.
\item Die Vereine können sich um die Ausrichtung einzelner Turniere -- ggf. Klassen oder Gruppen -- bewerben und können für die Durchführung eine
Aufwandsentschädigung gemäß der Finanzordnung des BSV erhalten.
\item Anmeldungen von Spielern und Mannschaften zu Turnieren des BSV sind grundsätzlich von den Vereinen vorzunehmen. Abweichungen davon können
in den Turnierausschreibungen zugelassen werden. Spieler oder Mannschaften, die nicht entsprechend der Ausschreibung angemeldet worden sind,
dürfen am Turnier nicht teilnehmen. Der zuständige Turnierleiter kann Ausnahmen zulassen. Diese Spieler oder Mannschaften zahlen vor Turnierbeginn die in der
Ausschreibung festgelegte Bearbeitungsgebühr und das Reuegeld.

\end{enumerate}

\subsection{Allgemeine Spielberechtigung}

\begin{enumerate}
\item An geschlossenen Turnieren dürfen die nach Mitgliederverwaltungsordnung aktiven Mitglieder der dem BSV angeschlossenen Vereine teilnehmen.
\item Der Landesspielleiter hat für alle Turniere unter seiner Verantwortung das ausdrückliche Recht, Spielgenehmigungen zu erteilen.
\item Zu Turnieren des Deutschen Schachbundes (DSB) oder der NDLV qualifizieren sich die bestplatzierten aktiven Mitglieder (Spieler oder Mannschaften),
die die Qualifikation nach der entsprechenden Turnierordnung wahrnehmen können.
\item Die Bildung von Spielgemeinschaften unter BSV-Vereinen ist zulässig.
\item Jugendliche, die am 1. Januar des Jahres, in dem die BMM beginnt, das 20. Lebensjahr noch nicht vollendet haben,
können mit einer passiven Spielberechtigung in der BMM gemeldet und eingesetzt werden.
\end{enumerate}

\subsection{Einspruch und Widerspruch}

\begin{enumerate}
\item Gegen Maßnahmen und Entscheidungen im Rahmen des Spielbetriebes kann in Textform beim Turnierleiter mit einer Frist von sieben Tagen Einspruch eingelegt werden.
Dieser entscheidet erstinstanzlich auch selbst, wenn die Entscheidung von ihm getroffen wurde. Eine Entscheidung soll innerhalb von 14 Tagen getroffen werden.
\item Gegen Entscheidungen des Turnierleiters kann Widerspruch beim Landesspielleiter mit einer Frist von sieben Tagen bei gleichzeitiger Zahlung einer Gebühr
in Höhe von 25 € auf das Konto des BSV erhoben werden.
\item Innerhalb von vierzehn Tagen sind den Betroffenen Bescheide in Textform unter Hinweis auf den weiteren Rechtsweg zuzusenden.
\item Die Fristen beginnen bei Wettkämpfen mit dem Spieltag, ansonsten mit dem Tag des Erhalts der Entscheidung / des Protestes.
\item Einspruch und Widerspruch müssen den Sachverhalt, einen Antrag und die Begründung in verständlicher Form enthalten.
\item Einspruch und Widerspruch können bis zu einer Entscheidung jederzeit zurückgezogen werden. Die Gebühren werden ggf. zurückerstattet.
\item Gegen Entscheidungen des Landesspielleiters können die Betroffenen gemäß §9 (2) der Verbandssatzung den Vermittlungsausschuss anrufen.
\end{enumerate}

\subsection{Ordnungsmaßnahmen}

\begin{enumerate}
\item Eingesetzte Schiedsrichter können Maßnahmen nach Artikel 12.9 der FIDE-Schachregeln ergreifen.
\item Zuständige Turnierleiter und der Landesspielleiter können zusätzlich gegenüber Einzelspielern und Mannschaften folgende Maßnahmen ergreifen:
  \begin{enumerate}[label=\alph*.]
  \item Abzug von Mannschafts- oder Brettpunkten
  \item Erhöhung der Mannschafts- oder Brettpunkte des Gegners
  \item Neuansetzung des Wettkampfs
  \end{enumerate}
\item Spieler (in Einzelturnieren) oder Mannschaften, die unentschuldigt zu einer Runde fehlen, werden aus dem Turnier genommen, steigen grundsätzlich in die nächstniedere
Spielklasse ab und zahlen das Reuegeld in Höhe von 50 €, das allerdings durch die Turnierausschreibung abgemildert werden kann. Ein Fehlen
in der letzten Runde in der BEM muss mit der Anmeldung zum Turnier angezeigt und durch die Turnierleitung genehmigt werden. Ein Verursachen eines kampflosen Punktes
wird nach §16 (5, 9) wie in der BMM geahndet.  % REFERENZ %
\item Spieler (in Einzelturnieren) oder Mannschaften werden ferner aus dem Turnier genommen, wenn sie:
  \begin{enumerate}[label=\alph*.]
  \item an zwei Tagen entschuldigt fehlen und trotzdem kampflose Punkte verursacht haben
  \item an drei Tagen nicht mitgespielt haben (gewonnene kampflose Punkte nicht mitgerechnet)
  \end{enumerate}
Der Turnierleiter kann in begründeten Fällen anders entscheiden.
\item Angemeldete Spieler oder Mannschaften, die nach Ablauf der Wartezeit am Turnierort erscheinen und Spieler oder Mannschaften, die das Turnier nicht ordnungsgemäß
beenden, zahlen ein Reuegeld gemäß der Turnierausschreibung.
\item Für Geldbußen, Geldstrafen und Reuegelder, die gegen Spieler oder Mannschaften gemäß dieser Turnierordnung verhängt werden, haften die Vereine dieser Spieler
oder Mannschaften gegenüber dem BSV.
\end{enumerate}

\section{Berliner Einzelmeisterschaft}

\subsection{Allgemeine Festlegungen}

\begin{enumerate}
\item Die Berliner Einzelmeisterschaft besteht aus Klassenturnieren und dem Qualifikationsturnier (QT). 
Die Klassen M, A, B, C und das QT sind geschlossene Turniere, die Klassen D sind offene Turniere.
\item
  \begin{enumerate}[label=\alph*.]
  \item Qualifikation. Die jeweiligen Klassenberechtigungen ergeben sich aus den Platzierungen in den Klassen- und Qualifikationsturnieren der letzten Jahre
  sowie im QT des aktuellen Jahres.
  \item Teilnahmeberechtigung. Ein Spieler kann in einem Jahr nur an einem Turnier der Klassen M bis C teilnehmen.
  \item Herabstufung bei Inaktivität. Nimmt ein vorberechtigter Spieler nicht nach dem Erlangen einer Qualifikation an der BEM teil, wird er nach drei Jahren um eine Klasse
  herabgestuft. Dabei werden auf Antrag Jahre nicht berücksichtigt, in denen der Spieler durch die Teilnahme an offiziellen Meisterschaften des DSB, der
  Deutschen Schachjugend (DSJ) oder der FIDE verhindert ist, seine Berechtigung für eine Spielklasse wahrzunehmen. Spieler, die sechs Jahre inaktiv sind, werden aus der
  Klassenberechtigungsliste gestrichen.
  \item Qualifikation über andere Wettbewerbe. Der Berliner Einzelpokalsieger erhält 
  die Berechtigung, im folgenden Jahr an der M-Klass teilzunehmen.
  Der Berliner Jugend-Einzelmeister U18 erhält auf Antrag dei Brechtigung, im aktuellen Jahr an der M-Klasse teilzunehmen.
  \end{enumerate}
\item Freiplatz. Wer noch keine Klassenberechtigung besitzt oder diese nicht durch einen in den letzten drei Jahren erspielten Abstieg hat, kann entsprechend
seiner Spielstärke den einzelnen Klassen für das beantragte Jahr zugeordnet werden. Dabei gelten folgende Grundsätze:

\begin{tabular}{c | c} 
DWZ / ELO & Klasse \\
\hline
$\ge 2200$ & M \\
$\ge 2000$ & A \\
$\ge 1800$ & B \\
$\ge 1600$ & C \\
\end{tabular}

Der Landesspielleiter kann Spieler auf begründeten Antrag in eine höhere Klasse einstufen.

\end{enumerate}

\subsection{Die Meisterklasse (M-Klasse)}

\begin{enumerate}
\item Die M-Klasse wird jährlich ausgetragen.
\item Der Sieger erhält den Titel: "`Berliner Meister (Jahr)"'.
\item Die M-Klasse wird bei bis zu zehn Teilnehmern als Rundenturnier ausgetragen. Bei mehr als zehn Teilnehmern wird Schweizer System mit maximal neun Runden gespielt.
\item Die Bedenkzeit beträgt je Spieler 100 min für 40 Züge zzgl. 30 min für den Rest der Partie sowie 30 s pro Zug vom ersten Zug an.
\item Pro angefangene fünf Teilnehmer steigt ein Spieler ab. Titelträger (GM, IM, FM) mit DWZ / ELO $\ge 2300$ und Titelträgerinnen (WGM, WIM, WFM) mit
DWZ / ELO $\ge 2100$ müssen nicht absteigen.
\item Bei ungerader Teilnehmeranzahl darf der Turnierleiter einen Spieler nachnominieren.
\item Die weiteren Modalitäten regelt die Turnierausschreibung.
\end{enumerate}

\subsection{Die Klassenturniere A--D}

\begin{enumerate}
\item Die Klassenturniere werden jährlich als Rundenturniere in mehreren Gruppen ausgetragen und beginnen nach dem QT.
\item Die Teilnehmerzahl je Gruppe soll 8 -- 12 Spieler betragen. Die genaue Anzahl der Spieler in einer Gruppe bestimmt sich nach der Gesamtteilnehmerzahl
einer Klasse und den Durchführungsangeboten der Vereine.
\item Hat ein Spieler mindestens eine Partie absolviert und tritt dann aus dem Turnier zurück, erhält jeder seiner Gegner, der das Turnier ohne Rücktritt beendet hat,
als Ergebnis gegen diesen Spieler einen kampflosen Punkt, wenn er nicht die entsprechende Partie gewonnen hat.
\item Die Bedenkzeit je Spieler kann betragen:
  \begin{itemize}
  \item 120 min für 40 Züge zzgl. 30 min für den Rest der Partie.
  \end{itemize}
  Beim Einsatz elektronischer Uhren:
  \begin{itemize}
  \item 90 min für 40 Züge zzgl. 15 min für den Rest der Partie sowie 30 s pro Zug vom ersten Zug an.
  \item 90 min für 40 Züge zzgl. 30 min für den Rest der Partie sowie 30 s pro Zug vom ersten Zug an.
  \item 100 min für 40 Züge zzgl. 30 min für den Rest der Partie sowie 30 s pro Zug vom ersten Zug an.
  \end{itemize}
\item Aufstieg
  \begin{itemize}
  \item Bei 5 -- 7 Spielern steigt der erstplatzierte Spieler, in der D-Klasse bei 7 Spielern die beiden erstplatzierten Spieler jeder Gruppe in die nächsthöhere Klasse auf.
  \item Bei 8 -- 12 Spielern steigen die beiden erstplatzierten Spieler jeder Gruppe in die nächsthöhere Klasse auf.
  \item Bei 13 -- 15 Spielern steigen die drei erstplatzierten Spieler jeder Gruppe in die nächsthöhere Klasse auf.
  \item Außerdem genügen 83 \% der möglichen Punkte, analog der 7,5/9 beim Qualifikationsturnier.
  \end{itemize}
\item Spieler, die weniger als 35 \% der möglichen Punkte erreicht haben, steigen ab.
\item Bei fünf oder sechs Teilnehmern kann auch doppelrundig gespielt werden.
\item Die weiteren Modalitäten regelt die Turnierausschreibung.
\end{enumerate}

\subsection{Das Qualifikationsturnier}

\begin{enumerate}
\item Das QT wird jährlich ausgetragen. Es werden neun Runden nach Schweizer System gespielt.
\item Folgende Qualifikationen können erworben werden:

\begin{tabular}{c | c} 
Erfolg & Klasse \\
\hline
Sieger & M \\
$7,5$ Punkte & M \\
$6,5$ Punkte & A \\
$5,5$ Punkte & B \\
$4$ Punkte & C
\end{tabular}

\item Spieler des BSV, die eine M-Klassenberechtigung haben, dürfen nicht am QT teilnehmen.
\item Die Bedenkzeit beträgt je Spieler 100 min für 40 Züge zzgl. 30 min für den Rest der Partie sowie 30 s pro Zug vom ersten Zug an.
\item Die weiteren Modalitäten regelt die Turnierausschreibung.
\end{enumerate}

\section{Berliner Mannschaftsmeisterschaft}

\subsection{Allgemeine Festlegungen}

\begin{enumerate}
\item Die Berliner Mannschaftsmeisterschaft (BMM) wird grundsätzlich im Zeitraum zwischen dem 1. September eines Jahres und dem 31. Mai des Folgejahres ausgetragen.
\item Es werden Rundenturniere in Ligen und Staffeln gespielt.
\item Der Sieger der Landesliga erhält den Titel: "`Berliner Mannschaftsmeister (Jahr)"'.
\item Die Wettkämpfe der BMM beginnen sonntags um 9:00 Uhr.
\item Die Bedenkzeit beträgt
  \begin{itemize}
  \item in der Landesliga sowie in der Stadtliga je Spieler 100 min für 40 Züge zzgl. 30 min für den Rest der Partie sowie 30 s pro Zug vom ersten Zug an
  \item darunter je Spieler 120 min für 40 Züge zzgl. 30 min für den Rest der Partie.
  \end{itemize}
\item Das Vor- oder Nachspielen einzelner Partien ist nicht gestattet.
\item Die Heimmannschaft hat an den ungeraden Brettern Schwarz und an den geraden Weiß.
\item Der gastgebende Verein hat das für den Wettkampf erforderliche Spielmaterial in ausreichender Menge und Qualität zur Verfügung zu stellen.
Die Bretter sind zu nummerieren. Die Räumlichkeiten müssen eine störungsfreie Abwicklung des Wettkampfes ermöglichen, gut belüftet und ausreichend beheizt sein.
\item Während des Wettkampfes sollen die Spieler die Möglichkeit haben, warme und kalte nichtalkoholische Getränke zu erwerben.
\item Wenn eine Mannschaft einen Wettkampf aufgrund einer Überschneidung mit einer überregionalen Meisterschaft verschieben möchte, muss der Verein dies vier Wochen
vor Beginn des zu verschiebenden Wettkampfes beim Turnierleiter beantragen.
\end{enumerate}

\subsection{Schiedsrichter}

\begin{enumerate}
\item An jedem Wettkampfort ist mindestens ein Schiedsrichter einzusetzen. Für den Einsatz des Schiedsrichters ist die Heimmannschaft verantwortlich.
Der Einsatz von Schiedsrichtern durch den Turnierleiter der BMM hat ungeachtet obiger Festlegung immer Vorrang. Werden Runden der BMM zentral ausgetragen,
ist der Turnierleiter der BMM für den Schiedsrichtereinsatz zuständig.
\item Die eingesetzten Schiedsrichter sind den Mannschaften in geeigneter Form vor Beginn des Wettkampfes bekannt zu geben, notfalls zu erfragen.
\item Der Schiedsrichter muss sich spätestens 15 Minuten vor Wettkampfbeginn am Wettkampfort einfinden.
\item Der Schiedsrichter hat die Richtigkeit der abgegebenen Aufstellungen zu kontrollieren und ggf. korrigieren zu lassen.
\item Bei einem Streitfall entscheidet der Schiedsrichter. Dieser ist berechtigt, dazu seine eigene Partie zu unterbrechen (Uhren anhalten) und unmittelbar nach
Beilegung des Streitfalles wieder aufzunehmen.
\item Gegen Entscheidungen des Schiedsrichters ist der Einspruch entsprechend dieser Turnierordnung zulässig, besonders bei Entscheidungen nach Richtlinie III.4+III.5
der FIDE-Schachregeln. In der BMM (und auch in der BFL, welche in dieser TO nicht explizit geregelt wird) findet nur Richtlinie III.5 Anwendung.
\item Der gastgebende Verein hat den kompletten Spielbericht an den Turnierleiter zu melden. Die Ergebnismeldung muss ihm spätestens zwei Tage nach dem Wettkampf vorliegen.
Liegt eine Wochen nach dem Wettkampf keine Ergebnismeldung vor, hat die Mannschaft des Gastgebers kampflos verloren.
\item Die von den Vereinen eingesetzten Schiedsrichter müssen mindestens eine gültige Lizenz als Verbandsschiedsrichter besitzen. Diese dürfen entweder einen Wettkampf leiten und
selbst spielen oder mehrere Wettkämpfe leiten, dann aber nicht selbst spielen. RSR, NSR, FA und IA dürfen mehrere Wettkämpfe leiten, auch wenn sie selbst spielen.
Die Gültigkeit der Lizenz richtet sich nach der jeweiligen Ausbildungsordnung.
\end{enumerate}

\subsection{Klasseneinteilung}

\begin{enumerate}
\item Die BMM wird in Staffeln von jeweils zehn Mannschaften in folgenden Spielklassen durchgeführt:
  \begin{itemize}
  \item Landesliga (eine Staffel)
  \item Stadtliga (zwei Staffeln)
  \item darunter vier Staffeln pro Klasse
  \end{itemize}
In den beiden niedrigsten Klassen sind Abweichungen möglich.
\item Die Zuordnung der Mannschaften in die Staffeln jeder Klasse erfolgt nach den Mannschaftsmeldungen der Vereine mit dem Ziel, möglichst ausgeglichene Staffeln zu bilden.
Kriterium ist der Wertungszahl-Durchschnitt der gemeldeten Stammaufstellungen. Die 4. Stadtklasse wird nach regionalen Gesichtspunkten ausgelost.
\item Neu gemeldete Mannschaften beginnen grundsätzlich in der untersten Spielklasse. Der Spielausschuss kann auf Antrag Ausnahmen genehmigen.
\item Bei Zusammenschluss von Vereinen beziehungsweise der Bildung von Spielgemeinschaften gehen die Spielberechtigungen der einzelnen Vereine auf den neuen Verein bzw. die
Spielgemeinschaft über.
\item Es wird an acht Brettern, in der 4. Stadtklasse an sechs Brettern gespielt.
\end{enumerate}

\subsection{Meldungen und Ranglisten}

\begin{enumerate}
\item Eine Spielermeldung muss immer enthalten: Verein, Nachname, Vorname, Spielercode. Sie setzt eine Eintragung in die Liste beim Referenten für Mitgliederverwaltung voraus.
\item Die Vereine melden bis zum festgesetzten Termin ihre Mannschaften und deren Spieler. Für jede Mannschaft können bis zu 16 Spieler, für die unterste Mannschaft des Vereins
bis zu 32 Spieler gemeldet werden.
\item Nachmeldungen sind unter Einhaltung der unter Absatz 2 genannten Höchstzahlen bis zum Termin der sechsten Runde möglich.
Sie müssen dem Turnierleiter zwei Tage vor dem beabsichtigten Einsatz vorliegen. Nachgemeldete Spieler werden der Rangfolge hinten angefügt.
\item Jeder Spieler einer Mannschaft erhält eine Codezahl, die aus zwei Doppelziffern besteht. Die erste Doppelziffer kennzeichnet die Nummer der Mannschaft
und die zweite Doppelziffer die Nummer des Spielers.
\item Für jede Mannschaft sind alle Spieler außer den eigenen und den bis zu 16 ersten Spielern der ggf. nachfolgenden Mannschaft gesperrt.
\item Wird ein Spieler in einer BMM-Mannschaft aufgestellt, ist er für alle tieferen Mannschaften in dieser Runde gesperrt.
\item Stammspieler von Mannschaften überregionaler Spielklassen sind für die BMM gesperrt. Ersatzspieler von Mannschaften überregionaler Spielklassen können für die BMM nicht
nachgemeldet werden.
\item Nach einem überregionalen Einsatz ist ein Spieler für die nächste Runde der BMM gesperrt. Maßgeblich hierfür ist der Termin, an dem tatsächlich gespielt wird. 
\emph{Mehrere Einsätze an einem Wochenende gelten als nur ein Einsatz}. 
\item Wird ein Spieler an Nr. 17 oder 18 überregional aufgestellt, ist er nach einem überregionalen Einsatz nicht gesperrt für die BMM. Allerdings gilt Abs. 8 letzter Satz.
\item Nach drei Einsätzen in höheren (auch überregionalen) Mannschaften ist ein Spieler für die verbleibenden Runden der tieferen Mannschaft gesperrt.
\end{enumerate}

\subsection{Mannschaftsaufstellungen und Aufnahme des Wettkampfes}

\begin{enumerate}
\item Der Mannschaftsleiter hat bis spätestens zehn Minuten vor dem festgesetzten Wettkampfbeginn die Mannschaftsaufstellung beim Schiedsrichter abzugeben.
Die Aufstellung muss die acht Spieler (in der 4. Stadtklasse sechs Spieler) mit Codenummern, Namen und Vornamen und die Unterschrift des Mannschaftsleiters
enthalten. Freilassen eines Brettes ohne Namensnennung gilt als unvollständige Aufstellung und wird nach 3.7 (2.) geahndet. % REFERENZ %
Eine Mannschaft ist nur dann richtig aufgestellt, wenn die Code-Zahlen der Spieler von Brett 1 zu Brett 8 größer werden.
\item Die korrekten Aufstellungen sind mit Wettkampfbeginn auszulegen und können danach nicht
mehr geändert werden.
\item Der Mannschaftskampf darf von einer Mannschaft aufgenommen werden, wenn die Aufstellung
abgegeben wurde und mindestens vier Spieler (in der 4. Stadtklasse drei Spieler) anwesend sind.
\item Wird ein Spieler ohne Mitgliedsmeldung eingesetzt, verliert er kampflos.
\item Ist ein aufgestellter Spieler gesperrt oder sitzt zu niedrig, verliert er kampflos.
\item Durch vertauschte Bretter verliert nur der dadurch zu tief sitzende getauschte Spieler kampflos.
\item Tritt eine Mannschaft nicht an, erhält sie keine Punkte. Die gegnerische Mannschaft erhält zwei Mannschaftspunkte und die auf halbe Punkte aufgerundete
Gewinnerwartung an Brettpunkten (mindestens die Hälfte des Maximums). Dabei wird an jedem Brett jeder Mannschaft der DWZ-Durchschnitt der bisher aufgestellten
Spieler herangezogen und über alle Bretter summiert, bevor gerundet wird. In der ersten Runde werden jeweils die besten acht spielberechtigten Spieler herangezogen.
\end{enumerate}

\subsection{Auf- und Abstieg}

\begin{enumerate}
\item Über die Platzierung innerhalb der Staffel entscheidet:
  \begin{enumerate}[label=\alph*.]
  \item die Summe der Mannschaftspunkte
  \item die Summe der Brettpunkte
  \item der direkte Vergleich
  \item die Berliner Wertung aus dem direkten Vergleich
  \item die Berliner Wertung aus allen Kämpfen
  \item das Los
  \end{enumerate}
\item Der Rang von Mannschaften verschiedener Staffeln einer Klasse ergibt sich aus:
  \begin{enumerate}[label=\alph*.]
  \item Platzierung (bei Abstieg von hinten beginnend)
  \item Summe der Mannschaftspunkte
  \item Summe der Brettpunkte
  \item das Los
  \end{enumerate}
\item Bei abweichender Rundenzahl ist die Wertung ins Verhältnis zu setzen.
\item Aufsteiger in die Oberliga ist die bestplatzierte aufstiegsberechtigte Mannschaft der Landesliga.
\item Aus jeder Staffel steigen die beiden letztplatzierten Mannschaften ab, sofern sich entsprechende Aufsteiger finden.
Steigen mehrere Mannschaften aus der Oberliga ab, erhöht sich die Anzahl der Absteiger jeder Klasse entsprechend.
\item Durch Abstieg oder Rückzug frei gewordene Plätze werden von Aufsteigern der nächstniederen Klasse besetzt.

\end{enumerate}

\subsection{Gebühren}

\begin{enumerate}
\item Unvollständige oder verspätete Mannschaftsmeldung: \hfill 15 €
\item Nichtantritt einer Mannschaft: \hfill 50 €
\item Einsatz eines Nichtmitglieds: \hfill 50 €
\item Einsatz eines gesperrten Spielers: \hfill 20 €
\item Freies Brett (ggf. zusätzlich zu Abs. 4; außer letzte Mannschaft, wenn sie in der 3. oder 4. Stadtklasse spielt): \hfill 10 €
\item Unvollständige oder verspätete Ergebnismeldung: \hfill 5 €
\item Fehlen eines Schiedsrichters mit gültiger Lizenz nach §11 (7), je Spieltag für den Heimverein: \hfill 20 € % REFERENZ %
\item Bei begründetem Antrag innerhalb von sieben Tagen nach dem Spieltag kann einmalig je Saison die Gebühr nach Abs. 7 erlassen werden.
\item Für die letzten beiden Runden der BMM verdoppeln sich die Gebühren. Diese Regelung findet keine Anwendung auf Abs. 7.
\end{enumerate}


\end{document}