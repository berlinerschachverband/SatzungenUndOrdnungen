\documentclass[fontsize=12pt, paper=a4, ngerman]{article}
\usepackage[ngerman]{babel}
\usepackage[utf8]{inputenc}
\usepackage[T1,EU1]{fontenc}

\usepackage{lmodern}
\usepackage[left=3cm, right=2.5cm, top=3cm, bottom=3cm]{geometry}

\usepackage{enumitem}

\usepackage[hidelinks]{hyperref}
\hypersetup{
  linktoc=all
}

\setlength{\parindent}{0em}

\begin{document}

\author{ Berliner Schachverband e.V. } \title{Finanzordnung} \date{\today} \maketitle

\tableofcontents

\newpage

\section{Ziele und Grundsätze}

\begin{enumerate}
\item Die Finanzordnung regelt die finanzielle Abwicklung aller Vorhaben des Berliner
Schachverbandes e. V. (im folgenden Verband genannt).

\item Alle Mittel sind im Sinne des § 1 der Satzung des Verbandes zu verwenden, wobei das
Prinzip der strengsten Sparsamkeit zu beachten ist.
\end{enumerate}

\section{Der Haushaltsplan}

\begin{enumerate}
\item Der Schatzmeister erstellt bis zum 15. Februar den Entwurf für das laufende
Geschäftsjahr. Die Grundlagen sind:
  \begin{enumerate}[label=\alph*)]
  \item Zuarbeiten der Präsidiumsmitglieder und der Referenten. In diesen Zuarbeiten sind
  die Vorhaben und Planungen für das folgende Geschäftsjahr zusammen mit einem
  Kostenvoranschlag darzustellen. Die Zuarbeiten sind jeweils bis zum 01.12. des
  Vorjahres dem Schatzmeister zuzustellen.
  \item Die tatsächlichen Einnahmen und Ausgaben des abgelaufenen Geschäftsjahres
  \item Vorgaben öffentlicher Zuschussgeber
  \end{enumerate}

\end{enumerate}  

\end{document}
