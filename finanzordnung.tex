\documentclass[fontsize=12pt, paper=a4, ngerman]{article}
\usepackage[ngerman]{babel}
\usepackage[utf8]{inputenc}
\usepackage[T1,EU1]{fontenc}

\usepackage{lmodern}
\usepackage[left=3cm, right=2.5cm, top=3cm, bottom=3cm]{geometry}

\usepackage{enumitem}

\usepackage[hidelinks]{hyperref}
\hypersetup{
  linktoc=all
}

\setlength{\parindent}{0em}

\begin{document}

\author{ Berliner Schachverband e.V. } \title{Finanzordnung} \date{\today} \maketitle

\tableofcontents

\newpage

\section{Ziele und Grundsätze}

\begin{enumerate}
\item Die Finanzordnung regelt die finanzielle Abwicklung aller Vorhaben des Berliner
Schachverbandes e. V. (im folgenden Verband genannt).

\item Alle Mittel sind im Sinne des § 1 der Satzung des Verbandes zu verwenden, wobei das
Prinzip der strengsten Sparsamkeit zu beachten ist.
\end{enumerate}

\section{Der Haushaltsplan}

\begin{enumerate}
\item Der Schatzmeister erstellt bis zum 15. Februar den Entwurf für das laufende
Geschäftsjahr. Die Grundlagen sind:
  \begin{enumerate}[label=\alph*)]
  \item Zuarbeiten der Präsidiumsmitglieder und der Referenten. In diesen Zuarbeiten sind
  die Vorhaben und Planungen für das folgende Geschäftsjahr zusammen mit einem
  Kostenvoranschlag darzustellen. Die Zuarbeiten sind jeweils bis zum 01.12. des
  Vorjahres dem Schatzmeister zuzustellen.
  \item Die tatsächlichen Einnahmen und Ausgaben des abgelaufenen Geschäftsjahres
  \item Vorgaben öffentlicher Zuschussgeber
  \end{enumerate}
\item Jedem Mitglied des Präsidiums und jedem Referenten ist im Haushaltsplan eine eigene
Kostenstelle zuzuweisen.
\item Der Haushaltsplan ist im Februar des Geschäftsjahres vom Präsidium zu beraten und zu
genehmigen sowie dem Verbandstag zur Annahme vorzulegen.
\item Der vom Verbandstag beschlossene Haushaltsplan ist für die Organe des Verbandes und
die Referenten grundsätzlich bindend. Das Präsidium kann im Rahmen des
Haushaltsplans Umverteilungen in den einzelnen Posten beschließen.
\item Das Präsidium kann bei Notwendigkeit einen Nachtragshaushalt beschließen. Über die
Gründe hat das Präsidium die Vereine unverzüglich zu informieren. Der Nachtragshaushalt
ist den Vereinen in geeigneter Form bekannt zu machen.
\end{enumerate}  

\section{Kassenführung und Jahresabschluss}
\begin{enumerate}
\item Der Schatzmeister hat über die vereinnahmten Beträge und deren Verwendung genau
Buch zu führen. Alle Einnahmen und Ausgaben müssen anhand von Belegen nachweisbar
sein.
\item Der Schatzmeister erstellt bis zum 15. Februar des Folgejahres die Abschlussrechnung
des abgelaufenen Geschäftsjahres. Diese legt er im Februar dem Präsidium zur Beratung
und Genehmigung sowie den Rechnungsprüfern zur Prüfung vor. Das Präsidium legt dem
Verbandstag die Abschlussrechnung zur Annahme vor.
\end{enumerate}

\section{Verwendung / Abrechnung der Mittel}
\begin{enumerate}
\item Aus den Einnahmen des Verbandes sind zu bestreiten:
  \begin{enumerate}[label=-]
  \item Kosten der Veranstaltungen des Verbandes
  \item Startgelder offizieller überregionaler Mannschaftsmeisterschaften mit Ausnahme der 1. Schachbundesliga (allg. und Frauen)
  \item Zuschüsse für die Teilnahme an Turnieren, Lehrgängen, Begegnungen und ähnlichem sowie für Startgelder
  offizieller überregionaler Meisterschaften
  \item allgemeine Geschäftskosten
  \item Auslagen des Präsidiums und der Organe des Verbandes
  \item Auslagen der Delegierten des Verbandes anlässlich von Tagungen übergeordneter Organisationen, soweit diese die Auslagen
  nicht erstatten
  \end{enumerate}
\item Die Erstattungen oder Zuschüsse müssen dieser Finanzordnung entsprechen, begründet
und eindeutig belegt werden. Zweckgebundene Mittel dürfen nur für die geplanten
Vorhaben eingesetzt werden.
\item Für jede Veranstaltung des Verbandes ist ein gesonderter Finanzplan
(Einnahmen/Ausgaben) aufzustellen und spätestens 14 Tage vor Beginn der Veranstaltung
dem Schatzmeister zu übergeben. Auslagen und anfallende Kosten werden grundsätzlich
nur auf dieser Grundlage erstattet. Die Verantwortung trägt das jeweilige Mitglied des
Präsidiums bzw. der zuständige Referent.
\item Die Abrechnung hat innerhalb einer Frist von 30 Tagen nach Beendigung der Veranstaltung
zu erfolgen. Später eingereichte Abrechnungen sind nur in begründeten Ausnahmefällen
zulässig. Die Verantwortung regelt sich nach Punkt 4.3. Die Referenten des Verbandes
reichen ihre Abrechnung über das jeweils zuständige Präsidiumsmitglied ein, das die Freigabe bestätigen muss.
\item Für die Mitgliedsvereine des Verbandes gelten die vorstehenden Festlegungen des § 4 entsprechend.
\end{enumerate}

\section{Zuwendungen}
\begin{enumerate}
\item Entsprechend der Satzung §1(8) erhalten ehrenamtliche Funktionsträger für die in der
Satzung vorgesehenen Aufgaben keine Zuwendungen.
\item Die entsprechend der Satzung § 3(1) in den Organen des Verbandes tätigen
Ehrenamtsträger können pro Jahr, ggf. anteilig, folgende Aufwandsentschädigungen
erhalten:
  \begin{enumerate}[label=-]
  \item Präsident/in \hfill 500 €
  \item Vizepräsident/in \hfill 300 €
  \item Landesspielleiter/in \hfill 300 €
  \item Schatzmeister/in \hfill 300 €
  \item Landesjugendwart/in \hfill 300 €
  \item Referent/in für Öffentlichkeitsarbeit \hfill 100 €
  \item Referent/in für Mitgliederverwaltung \hfill 100 €
  \item Referent/in für DWZ \hfill 150 €
  \item Referent/in für Frauenschach \hfill 100 €
  \item Referent/in für Ausbildung \hfill 100 €
  \item Referent/in für Freizeit- und Breitensport \hfill 100 €
  \item Referent/in für Schulschach \hfill 100 €
  \item Referent/in für Seniorenschach \hfill 100 €
  \item Referent/in für Leistungssport \hfill 100 €
  \item Materialwart/in \hfill 100 €
  \item Beisitzer/in im Spielausschuss \hfill 100 €
  \item Beisitzer/in im Jugendausschuss \hfill 100 €
  \end{enumerate}
Bei Mehrfachfunktionen entscheidet das Präsidium über die Höhe der einzelnen
Aufwandsentschädigungen.
Das Präsidium kann im Rahmen der zur Verfügung stehenden Mittel des Verbandes weitere
Aufwandsentschädigungen beschließen.
\item Für Tätigkeiten, die über das Ehrenamt hinausgehen, können an Funktionsträger des
Verbandes und Mitglieder seiner Mitgliedsvereine (§2 (1.2)) folgende Zahlungen geleistet
werden:
  \begin{enumerate}
  \item Schiedsrichtereinsatz \\
  Schiedsrichtereinsätze Für alle Turniere des Verbandes sind ausgebildete und
  lizenzierte Schiedsrichter einzusetzen.
  Die Anzahl der einzusetzenden Schiedsrichter richtet sich nach der Anzahl
  der Turnierteilnehmer, es gilt der Grundsatz: je angefangene 50 Teilnehmer ein
  Schiedsrichter. Die Aufwandsentschädigungen richtet sind nach deren erreichter
  Qualifikation und der Einsatzzeit am jeweiligen Tag. \\
  \begin{tabular}{l r r} 
  Qualifikation & \hspace{2cm}bis 7h/Tag & \hspace{2cm}$>$ 7h/Tag \\
  \hline
  Nationaler Schiedsrichter (NSR)  & 25,- €/Tag & 30,- €/Tag \\
  Regionaler Schiedsrichter (RSR)  & 23,- €/Tag & 28,- €/Tag \\
  Verbandsschiedsrichter (VSR)     & 20,- €/Tag & 25,- €/Tag \\
  Turnierleiter (TL) - alte Lizenz & 20,- €/Tag & 25,- €/Tag \\
  Turnierhelfer (ohne Titel)       & 13,- €/Tag & 20,- €/Tag \\
  \end{tabular}
  \item Referent bei Ausbildungen des Verbandes, Trainer/Übungsleiter \\
  Für den Verband tätige Trainer, Übungsleiter und Referenten erhalten eine
  Aufwandsentschädigungen entsprechend ihrer Qualifikation: \\
    \begin{enumerate}[label=-]
    \item A-Trainer \hfill 35,- €/h
    \item B-Trainer \hfill 30,- €/h
    \item C-Trainer Leistungssport \hfill 25,- €/h
    \item C-Trainer Breitensport \hfill 25,- €/h
    \item Referenten \hfill 15,- €/h
    \end{enumerate}
  \item Das Präsidium kann für Mitglieder des BSV abweichende Festlegungen treffen.
  \end{enumerate}
  \item Für den Verband tätige Personen, die nicht mittelbar dem Verband angehören, gelten
  für die Zahlung von Honoraren u.ä. die vorstehend genannten Grundsätze und Zahlen. Das
  Präsidium kann davon abweichende Festlegungen treffen.
\end{enumerate}

\section{Reisekosten}
\begin{enumerate}
  \item Reisekosten werden in Anwendung der Reisekostenordnung des Landessportbundes
  Berlin vergütet, soweit diese Kosten nicht von dritter Seite erstattet werden.
  \item Für die Antragstellung und Abrechnung gilt Punkt 4 entsprechend. Es ist die jeweils
  günstigste Reisemöglichkeit zu nutzen. Das gilt insbesondere bei mehreren Teilnehmern.
  In besonderen Fällen kann das Präsidium auf Antrag davon abweichende Festlegungen treffen.
  \item Übernachtungen und Tagesspesen werden grundsätzlich nur bei Veranstaltungen des
  Deutschen Schachbundes nach dessen Regeln gewährt. In besonderen Fällen kann das
  Präsidium davon abweichende Festlegungen treffen.
\end{enumerate}

\section{Schlussbestimmungen}
\begin{enumerate}
  \item Die Schachjugend des Verbandes arbeitet im Rahmen der zugewiesenen finanziellen
  Mittel im Haushaltsplan eigenständig nach einer von ihr zu erlassenen
  Jugendfinanzordnung. Grundlage bilden die Festlegungen dieser Finanzordnung. Die
  Abrechnungen und Zahlungen erfolgen jedoch ausschließlich über den Schatzmeister des
  Verbandes.
  \item Der Landesspielleiter sowie die Referenten für Frauen- und Seniorenschach können im
  Rahmen ihrer Haushaltsmittel für die in ihrem Verantwortungsbereich durchzuführenden
  Turniere Startgelder einnehmen und Preisgelder ausschütten. Die Abrechnung erfolgt unter
  Beachtung Punkt 4 dieser Finanzordnung ausschließlich über den Schatzmeister des Verbandes.
  \item Das Präsidium des Verbandes hat in jedem Fall das Recht, die Verwendung der Mittel
  zu prüfen und bei Notwendigkeit Korrekturen vorzunehmen.
  \item Salvatorische Klausel \\
  Sollte eine oder mehrere der vorliegenden Bestimmungen ungültig sein oder werden so
  behalten die anderen ihre Gültigkeit.
\end{enumerate}

Die Finanzordnung des Berliner Schachverbandes e. V. wurde am 18. April 2011 vom Präsidium
einstimmig beschlossen, in Kraft gesetzt und mehrfach geändert, zuletzt am 15. Juli 2015 durch einstimmigen Beschluss des Präsidiums.

\end{document}
