\documentclass[fontsize=12pt, paper=a4, ngerman]{article}
\usepackage[ngerman]{babel}
\usepackage[utf8]{inputenc}
\usepackage[T1,EU1]{fontenc}

\usepackage{lmodern}
\usepackage[left=3cm, right=2.5cm, top=3cm, bottom=3cm]{geometry}

\usepackage{enumitem}

\usepackage[hidelinks]{hyperref}
\hypersetup{
  linktoc=all
}

\setlength{\parindent}{0em}

\begin{document}

\author{ Berliner Schachverband e.V. } \title{Finanzordnung} \date{\today} \maketitle

\tableofcontents

\newpage

\section{Ziele und Grundsätze}

\begin{enumerate}
\item Die Finanzordnung regelt die finanzielle Abwicklung aller Vorhaben des Berliner
Schachverbandes e. V. (im folgenden Verband genannt).

\item Alle Mittel sind im Sinne des § 1 der Satzung des Verbandes zu verwenden, wobei das
Prinzip der strengsten Sparsamkeit zu beachten ist.
\end{enumerate}

\section{Der Haushaltsplan}

\begin{enumerate}
\item Der Schatzmeister erstellt bis zum 15. Februar den Entwurf für das laufende
Geschäftsjahr. Die Grundlagen sind:
  \begin{enumerate}[label=\alph*)]
  \item Zuarbeiten der Präsidiumsmitglieder und der Referenten. In diesen Zuarbeiten sind
  die Vorhaben und Planungen für das folgende Geschäftsjahr zusammen mit einem
  Kostenvoranschlag darzustellen. Die Zuarbeiten sind jeweils bis zum 01.12. des
  Vorjahres dem Schatzmeister zuzustellen.
  \item Die tatsächlichen Einnahmen und Ausgaben des abgelaufenen Geschäftsjahres
  \item Vorgaben öffentlicher Zuschussgeber
  \end{enumerate}
\item Jedem Mitglied des Präsidiums und jedem Referenten ist im Haushaltsplan eine eigene
Kostenstelle zuzuweisen.
\item Der Haushaltsplan ist im Februar des Geschäftsjahres vom Präsidium zu beraten und zu
genehmigen sowie dem Verbandstag zur Annahme vorzulegen.
\item Der vom Verbandstag beschlossene Haushaltsplan ist für die Organe des Verbandes und
die Referenten grundsätzlich bindend. Das Präsidium kann im Rahmen des
Haushaltsplans Umverteilungen in den einzelnen Posten beschließen.
\item Das Präsidium kann bei Notwendigkeit einen Nachtragshaushalt beschließen. Über die
Gründe hat das Präsidium die Vereine unverzüglich zu informieren. Der Nachtragshaushalt
ist den Vereinen in geeigneter Form bekannt zu machen.
\end{enumerate}  

\section{Kassenführung und Jahresabschluss}
\begin{enumerate}
\item Der Schatzmeister hat über die vereinnahmten Beträge und deren Verwendung genau
Buch zu führen. Alle Einnahmen und Ausgaben müssen anhand von Belegen nachweisbar
sein.
\item Der Schatzmeister erstellt bis zum 15. Februar des Folgejahres die Abschlussrechnung
des abgelaufenen Geschäftsjahres. Diese legt er im Februar dem Präsidium zur Beratung
und Genehmigung sowie den Rechnungsprüfern zur Prüfung vor. Das Präsidium legt dem
Verbandstag die Abschlussrechnung zur Annahme vor.
\end{enumerate}

\section{Verwendung / Abrechnung der Mittel}
\begin{enumerate}
\item Aus den Einnahmen des Verbandes sind zu bestreiten:
  \begin{enumerate}[label=-]
  \item Kosten der Veranstaltungen des Verbandes
  \item Startgelder offizieller überregionaler Mannschaftsmeisterschaften mit Ausnahme der 1. Schachbundesliga (allg. und Frauen)
  \item Zuschüsse für die Teilnahme an Turnieren, Lehrgängen, Begegnungen und ähnlichem sowie für Startgelder
  offizieller überregionaler Meisterschaften
  \item allgemeine Geschäftskosten
  \item Auslagen des Präsidiums und der Organe des Verbandes
  \item Auslagen der Delegierten des Verbandes anlässlich von Tagungen übergeordneter Organisationen, soweit diese die Auslagen
  nicht erstatten
  \end{enumerate}
\item Die Erstattungen oder Zuschüsse müssen dieser Finanzordnung entsprechen, begründet
und eindeutig belegt werden. Zweckgebundene Mittel dürfen nur für die geplanten
Vorhaben eingesetzt werden.
\item Für jede Veranstaltung des Verbandes ist ein gesonderter Finanzplan
(Einnahmen/Ausgaben) aufzustellen und spätestens 14 Tage vor Beginn der Veranstaltung
dem Schatzmeister zu übergeben. Auslagen und anfallende Kosten werden grundsätzlich
nur auf dieser Grundlage erstattet. Die Verantwortung trägt das jeweilige Mitglied des
Präsidiums bzw. der zuständige Referent.
\item Die Abrechnung hat innerhalb einer Frist von 30 Tagen nach Beendigung der Veranstaltung
zu erfolgen. Später eingereichte Abrechnungen sind nur in begründeten Ausnahmefällen
zulässig. Die Verantwortung regelt sich nach Punkt 4.3. Die Referenten des Verbandes
reichen ihre Abrechnung über das jeweils zuständige Präsidiumsmitglied ein, das die Freigabe bestätigen muss.
\item Für die Mitgliedsvereine des Verbandes gelten die vorstehenden Festlegungen des § 4 entsprechend.
\end{enumerate}

\section{Zuwendungen}
\begin{enumerate}
\item Entsprechend der Satzung §1(8) erhalten ehrenamtliche Funktionsträger für die in der
Satzung vorgesehenen Aufgaben keine Zuwendungen.
\item Die entsprechend der Satzung § 3(1) in den Organen des Verbandes tätigen
Ehrenamtsträger können pro Jahr, ggf. anteilig, folgende Aufwandsentschädigungen
erhalten:
\end{enumerate}

\end{document}
